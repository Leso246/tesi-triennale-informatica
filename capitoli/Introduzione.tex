\chapter*{Introduzione}  
\addcontentsline{toc}{chapter}{Introduzione} 

Il campo del recruiting è in continua evoluzione, con una crescente necessità di strumenti 
che facilitino l'incontro tra domanda e offerta di lavoro in modo semplice ed efficiente. 
In questo contesto, la gestione dei colloqui di lavoro rappresenta una fase cruciale ma spesso 
complessa del processo di selezione. 
\\
L'obiettivo di questa esperienza di stage, svoltasi presso l'azienda Nesecom SRLS, 
è stato quello di integrare un sistema per la gestione dei meeting per i colloqui di lavoro 
all'interno di un sito in sviluppo, RisUma.
\\ 
RisUma, acronimo di RISorse UMAne, è pensato per essere una piattaforma di recruiting intuitiva, 
che consente agli utenti e alle aziende di trovare efficacemente le opportunità di lavoro e 
le figure professionali richieste, gestendo l'intero processo di selezione su un'unica piattaforma.
\\
Questo progetto è ancora nelle fasi iniziali del suo sviluppo, ma punta a eguagliare 
e superare le più famose piattaforme di recruiting grazie alle sue comode funzionalità.

\section*{Tecnologie utilizzate}
\addcontentsline{toc}{section}{Tecnologie utilizzate} 
Nel corso dello sviluppo del progetto, sono state impiegate diverse tecnologie 
moderne per garantire un'applicazione robusta, scalabile e facile da mantenere. 
Il progetto ha richiesto uno sviluppo full-stack, coinvolgendo sia il frontend che il backend.

\subsection*{Frontend} 
\addcontentsline{toc}{subsection}{Frontend}
Il frontend rappresenta la parte dell'applicazione con cui l'utente interagisce direttamente. 
È stato sviluppato utilizzando \textbf{React} insieme a \textbf{TypeScript}, una scelta che offre vantaggi 
significativi in termini di sicurezza e manutenibilità del codice. React è una libreria 
JavaScript per la creazione di interfacce utente dinamiche e reattive, mentre TypeScript 
aggiunge il supporto per i tipi statici, migliorando l'affidabilità del codice. Per gestire 
i pacchetti e le dipendenze, è stato utilizzato \textbf{npm} (Node Package Manager).

\begin{itemize}

    \item \textbf{React}: Utilizzato per costruire un'interfaccia utente moderna e reattiva, 
    con componenti riutilizzabili che facilitano lo sviluppo e la manutenzione del codice.

    \item \textbf{TypeScript}: Aggiunge tipi statici al linguaggio JavaScript, riducendo gli errori 
    e migliorando la qualità del codice.

    \item \textbf{MUI Theme (Material-UI)}: Utilizzato come template per il design del frontend. 
    Material-UI è una libreria di componenti React che implementa le linee guida del \textbf{Material 
    Design} di Google, fornendo componenti pre-stilizzati che assicurano un'interfaccia utente 
    coerente e professionale.
    
    \item \textbf{npm}: Gestore di pacchetti utilizzato per installare e gestire le dipendenze necessarie 
    per lo sviluppo con React. 

\end{itemize}

\subsection*{Backend} 
\addcontentsline{toc}{subsection}{Backend}
Il backend rappresenta la parte dell'applicazione che gestisce la logica di business, 
l'elaborazione dei dati e la comunicazione con il database. 
È responsabile del funzionamento lato server dell'applicazione, 
elaborando le richieste degli utenti e restituendo le risposte appropriate al frontend.
Il backend dell'applicazione è stato sviluppato utilizzando \textbf{Java 22} con il framework 
\textbf{Spring Boot}, che facilita la creazione di applicazioni stand-alone, 
production-ready con minima configurazione. Inoltre, l'utilizzo di \textbf{Maven}
come strumento di gestione delle dipendenze. Per la gestione della persistenza 
dei dati è stato utilizzato \textbf{Java Persistence API (JPA)}, che permette di interfacciarsi 
con il database in modo efficiente e strutturato. Il database utilizzato è \textbf{PostgreSQL 14}.

%bibliografia: https://it.wikipedia.org/wiki/Java_(linguaggio_di_programmazione)
\begin{itemize}
    \item \textbf{Java}: Linguaggio di programmazione ad alto livello orientato agli oggetti e 
    a tipizzazione statica, utilizzato per la sua robustezza.

    \item \textbf{Spring Boot}: Framework che riduce il tempo e lo sforzo necessari 
    per sviluppare applicazioni Java grazie alla sua configurazione automatica 
    e alla gestione delle dipendenze. DA RIVEDERE

    \item \textbf{Maven}: Strumento di build automation utilizzato per gestire il ciclo di vita del progetto, 
    le dipendenze e per facilitare l'integrazione continua. DA RIVEDERE

    \item \textbf{Java Persistence API (JPA)}: Utilizzato per la gestione della persistenza dei dati, 
    JPA offre un modo standard per mappare le classi Java agli oggetti del database, 
    semplificando notevolmente le operazioni di CRUD (Create, Read, Update, Delete).

    \item \textbf{PostgreSQL 14}: Sistema di gestione di database relazionale open source, 
    scelto per la sua affidabilità, scalabilità e le sue avanzate funzionalità.

\end{itemize}