\chapter{Approccio Proposto}
\section{Problema e Obiettivo}
Il problema principale affrontato riguarda:

\begin{itemize}
   \item Per le \textbf{aziende}: ottimizzazione dell'efficienza nella gestione dei meeting per i colloqui di lavoro, 
   fornendo uno strumento adeguato e completo di tutte le funzionalità necessarie. 
   L'obiettivo principale è migliorare l'esperienza dei recruiter, che potrebbero trovarsi a gestire un elevato numero di meeting.

    \item Per gli \textbf{utenti}: eliminare la necessità per i candidati di segnare manualmente gli appuntamenti, 
    offrendo loro uno strumento che ricordi in modo comodo e automatico tutti i colloqui programmati, facilitandone l'accesso.
 \end{itemize}

   \subsection{Caratteristiche dei Meeting}
   \label{sec:caratteristiche_meeting}
   Un ulteriore obiettivo riguarda le proprietà che devono caratterizzare un meeting:

      \begin{itemize}
         \item Non è richiesto avere un account per partecipare al meeting, ma sarà sufficiente inserire un nome a propria scelta al momento dell'accesso.
         
         \item Nessuno può accedere al meeting prima di 10 minuti rispetto all'orario di inizio prestabilito o dopo che sia trascorso l'orario di fine.
      \end{itemize}

\section{Metodologia}
Di seguito sono riportati i passi seguiti per la realizzazione di questo progetto.
Si vuole far notare che RisUma è un'idea del direttore aziendale, 
e pertanto non vi è un cliente con cui confrontarsi. Inoltre, poiché non sono state disponibili direttive scritte ma solo 
alcune indicazioni orali, tutte le scelte sono state prese a mia completa discrezione.

% Lista numerata
\begin{enumerate}
   \label{sec:scelta_del_servizo}
   \item \textbf{Servizio}: il primo passo è stato scegliere quale servizio di terze parti utilizzare per le chiamate online. 
   Si cercava un servizio che: 
   
      \begin{enumerate}
         \item avesse le \hyperref[sec:caratteristiche_meeting]{caratteristiche richieste}
         \item si integrasse bene con il progetto (API disponibili, compatibilità con le tecnologie utilizzate...)
         \item fosse facilmente usabile anche dagli utenti meno esperti
      \end{enumerate}
   
   Per tali motivi sono state prese in considerazione le piattaforme più note, quali Cisco Webex, Google Meet e Zoom.
   
   \item \textbf{Interfaccia utente}: è stato scelto il design della pagina frontend. 
   Basandosi anche sulle \hyperref[sec:euristiche_nielsen]{10 euristiche di Jakob Nielsen}, si sono perseguiti i seguenti obiettivi:

   % Lista  numerata interfaccia utente
   \begin{enumerate}
      \item Adattare il sistema al mondo reale. (Euristica \hyperref[sec:euristiche_nielsen2]{2})
      \item La pagina doveva essere facile da navigare, riducendo al minimo il carico cognitivo per l'utente. 
      (Euristica \hyperref[sec:euristiche_nielsen6]{6})
      \item  Optare per uno stile minimalista e intuitivo. (Euristica \hyperref[sec:euristiche_nielsen7]{7})
      \item Implementare funzionalità di feedback per guidare l'utente in caso di errori o problemi durante l'utilizzo della pagina.
      (Euristiche \hyperref[sec:euristiche_nielsen8]{8} e \hyperref[sec:euristiche_nielsen9]{9})
   \end{enumerate}

   \item \textbf{Creazione della tabella a database}: è stata progettata e creata la tabella nel database aziendale,
   includendo tutte le colonne necessarie,  al fine di garantire il corretto funzionamento della pagina web.

   \item \textbf{Integrazione delle API}: sono state integrate le API per consentire le operazioni CRUD sui meeting, 
   sviluppando contemporaneamente le parti di frontend e backend secondo il seguente ordine:

      \begin{enumerate}

         \item \textbf{Creazione}: implementazione della funzionalità per creare nuovi meeting, 
         gestendo i parametri necessari come data, ora e invitati.

         \item \textbf{Recupero}: realizzazione della logica per recuperare e visualizzare i meeting di un utente.

         \item \textbf{Modifica}: implementazione della possibilità dei modificare i parametri dei meeting già esistenti.

         \item \textbf{Eliminazione}: implementazione della funzionalità per annullare meeting programmati.
      \end{enumerate}

   Questa parte è stata cruciale nel corso della mia esperienza di stage
   in quanto ha costituito il nucleo essenziale della pagina che ho progettato e integrato.

   \item \textbf{Test}: Il sistema è stato sottoposto da parte mia e di alcuni colleghi a dei test per garantire che tutte 
   le funzionalità operassero correttamente.

   \item \textbf{Deploy}:  È stato eseguito un deploy su un server per testare le prestazioni in un ambiente
   più realistico rispetto a quello locale.
\end{enumerate}

\section{Possibili sfide}
Nel corso dello sviluppo del progetto sono emerse due principali sfide:
\begin{enumerate}

   \item \textbf{Consistenza tra i database}: assicurare la consistenza tra il database aziendale e quello del servizio 
   di terze parti durante le operazioni di creazione, modifica o eliminazione di un meeting. Una eventuale inconsistenza 
   porterebbe a gravi problemi legati allo svolgimento dei meeting, compromettendo l'affidabilità e le funzionalità 
   del sistema.

   \item \textbf{Creazione dell'interfaccia utente}: la realizzazione di un'interfaccia utente piacevole e funzionale
   ha rappresentato un'altra grande sfida. Nel corso dello sviluppo è stata più volte modificata sulla base dei feedback 
   dei colleghi.
\end{enumerate} 