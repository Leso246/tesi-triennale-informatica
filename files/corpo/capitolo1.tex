\chapter{Tecnologie utilizzate}
Nel corso dello sviluppo del progetto sono state impiegate diverse tecnologie 
moderne, per garantire un'applicazione robusta, scalabile e facilmente manutenibile. 
La parte del progetto a me assegnata ha richiesto uno sviluppo full-stack, 
coinvolgendo sia il frontend che il backend.

\section{Frontend} 
Il frontend rappresenta la parte dell'applicazione con cui l'utente interagisce direttamente. 
È stato sviluppato utilizzando \textbf{React} insieme a \textbf{TypeScript}, 
con l'obiettivo di garantire un'esperienza utente fluida e intuitiva.
\\
Per gestire i pacchetti e le dipendenze, è stato utilizzato \textbf{npm} (Node Package Manager).
Inoltre, il design dell'interfaccia utente è stato standardizzato 
utilizzando il \textbf{MUI Theme} di Material-UI.
\\
Per la comunicazione tra frontend e backend, è stata utilizzata la libreria \textbf{Axios}, 
che consente di effettuare richieste HTTP.


\begin{itemize}
    \item \textbf{React}: React è una libreria JavaScript open source, sviluppata da Facebook, 
    per la costruzione di interfacce utente. Utilizzando un approccio basato sui componenti, 
    React consente di creare interfacce utente modulari e riutilizzabili. La sua capacità di 
    aggiornare e mostrare efficientemente solo i componenti necessari in risposta ai cambiamenti 
    rende React particolarmente adatto per lo sviluppo di applicazioni interattive e 
    ad alte prestazioni. Inoltre, React permette di gestire lo stato delle applicazioni in modo 
    prevedibile e scalabile, facilitando lo sviluppo di applicazioni complesse. 
    \cite{ReactWikipedia}
 
    \item \textbf{TypeScript}: TypeScript è un linguaggio di programmazione open source sviluppato 
    da Microsoft, che estende JavaScript aggiungendo tipi statici. 
    L'adozione di TypeScript permette di ridurre significativamente gli errori
    durante lo sviluppo, grazie al controllo statico dei tipi che individua potenziali
    problemi prima ancora dell'esecuzione del codice. Inoltre, TypeScript facilita la manutenzione 
    del codice in progetti di grandi dimensioni, migliorando la leggibilità e la documentazione attraverso
    la tipizzazione esplicita. \cite{TypescriptWikipedia}
 
    \item \textbf{MUI Theme (Material-UI)}: Material-UI è una libreria di componenti React 
    che implementa le linee guida del \textbf{Material Design} di Google. Utilizzare 
    Material-UI permette di sviluppare interfacce utente coerenti e professionali 
    senza dover creare e stilizzare i componenti da zero. La libreria offre una vasta 
    gamma di componenti pre-stilizzati e altamente personalizzabili, che facilitano la 
    creazione di un design uniforme e accattivante. Inoltre, Material-UI supporta nativamente 
    la creazione di temi, consentendo di personalizzare l'aspetto dell'applicazione in modo centralizzato.

    \item \textbf{npm}: Node Package Manager è il gestore di pacchetti predefinito 
    per l'ecosistema JavaScript, utilizzato per installare e 
    gestire le dipendenze necessarie per lo sviluppo delle applicazioni. 
    Npm facilita l'integrazione di librerie e strumenti di terze parti, 
    permettendo agli sviluppatori di accedere rapidamente a un vasto repository 
    di pacchetti open source. Questo strumento è fondamentale per mantenere 
    aggiornate le dipendenze del progetto e per gestire le versioni delle librerie utilizzate.

    \item \textbf{Axios}:  Axios è una popolare libreria JavaScript per effettuare richieste HTTP dal browser e da Node.js. 
    Grazie alla sua interfaccia basata su Promises, Axios consente di inviare richieste al backend in modo semplice.
    Supporta tutte le operazioni HTTP principali, come GET, POST, PUT e DELETE, e offre funzionalità avanzate come 
    l'intercettazione delle richieste e delle risposte e la gestione automatica degli errori. 
    Queste caratteristiche lo rendono uno strumento potente e flessibile per la comunicazione tra frontend e backend. 
    \cite{Axios}

\end{itemize}

\section{Backend} 
Il backend rappresenta la parte dell'applicazione che gestisce la logica di business, 
l'elaborazione dei dati e la comunicazione con il database. 
È responsabile del funzionamento lato server dell'applicazione, 
elaborando le richieste degli utenti e restituendo le risposte appropriate al frontend.
\\
Il backend dell'applicazione è stato sviluppato utilizzando \textbf{Java} con il framework 
\textbf{Spring Boot}. Inoltre, viene utilizzato \textbf{Maven}
come strumento di gestione delle dipendenze. 
\\
Per la gestione della persistenza 
dei dati è stato utilizzato \textbf{Java Persistence API (JPA)}, che permette di interfacciarsi 
con il database in modo efficiente e strutturato. Il database utilizzato si basa su \textbf{PostgreSQL 14}.

\begin{itemize}
    \item \textbf{Java}: Java è un linguaggio di programmazione ad alto livello, 
    orientato agli oggetti e a tipizzazione statica, sviluppato da 
    Sun Microsystems (ora di proprietà di Oracle). È noto per la sua robustezza, 
    sicurezza e portabilità, grazie alla Java Virtual Machine (JVM) che permette di 
    eseguire il codice Java su qualsiasi piattaforma. Java è ampiamente utilizzato 
    nello sviluppo di applicazioni enterprise, sistemi embedded, applicazioni mobili e web. 
    La sua vasta ecosistema di librerie e strumenti lo rende una scelta ideale per lo sviluppo backend.
    \cite{JavaWikipedia}
 
    \item \textbf{Spring Boot}: Spring Boot è un framework open source 
    per lo sviluppo di applicazioni Java basato su Spring, che semplifica 
    il processo di creazione di app web e microservizi. 
    Spring Boot offre una configurazione automatica dei componenti, 
    riducendo significativamente il tempo necessario per avviare un progetto. Fornisce anche 
    una serie di funzionalità integrate, come la gestione della sicurezza e il logging 
    che facilitano lo sviluppo e la manutenzione delle applicazioni. \cite{MicrosoftSpringBoot}

    \item \textbf{Maven}: Maven è uno strumento di build automation e gestione 
    delle dipendenze per progetti Java sviluppato dalla Apache Software Foundation. 
    Utilizzando un modello basato su pom.xml  (Project Object Model), 
    Maven facilita la gestione del ciclo di vita del progetto, 
    dall'inizializzazione alla distribuzione. Consente di integrare facilmente le librerie 
    di terze parti e di gestire le versioni delle dipendenze, migliorando la coerenza e la 
    riproducibilità del build.
    \cite{MavenNexTre}

    \item \textbf{Java Persistence API (JPA)}: La Java Persistence API (JPA) è una specifica standard 
    per la gestione della persistenza dei dati nelle applicazioni Java. JPA offre un modo standardizzato
    per mappare le classi Java agli oggetti del database, permettendo di eseguire operazioni CRUD 
    (Create, Read, Update, Delete) in modo semplice e intuitivo.
    \\
    Inoltre fornisce un linguaggio per effettuare query SQL, chiamato \textbf{JPQL} 
    (Java Persistence Query Language), che è indipendente dal DBMS utilizzato.
    \\
    Utilizzando JPA, gli sviluppatori possono concentrarsi sulla logica di business senza 
    doversi preoccupare dei dettagli specifici dell'interazione con il database, 
    migliorando la produttività e la manutenibilità del codice.
    \cite{JpaVincenzoRacca}

    \item \textbf{PostgreSQL}: PostgreSQL è un sistema open source di database relazionale 
    a oggetti sviluppato da Michael Stonebraker. È famoso per la sua affidabilità e per 
    l'integrità dei dati, oltre alla community che continua a sostenerlo continuando
    a sviluppare nuove soluzioni.
    \cite{PostgreSQL}
    
\end{itemize}