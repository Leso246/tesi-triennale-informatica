\chapter{Struttura del progetto}
\section{Frontend}
Il frontend del progetto è stato strutturato in modo da massimizzare la leggibilità del codice e sfruttare al meglio 
la modularità offerta da React.

\begin{itemize}
    \item \textbf{Pagina principale}: la pagina principale costituisce il nucleo dell'interfaccia utente,
    funge da contenitore per i vari componenti che insieme andranno a formare la pagina visualizzata. 

    \item \textbf{Componenti}: ogni componente è progettato per gestire una specifica parte dell'interfaccia utente. 
    Questo favorisce la modularità di React, consentendo di riutilizzare facilmente le stesse interfacce in diverse parti dell'applicazione,
    oltre a facilitare la manutenzione del codice e a migliorarne la leggibilità. 

    \item \textbf{Metodi}: tutti i metodi utilizzati nella logica del frontend sono posti in file separati, suddivisi per semantica.
    Questa struttura migliora la leggibilità e la comprensione del codice.
    
    \item \textbf{useEffect}: uno \texttt{useEffect} è un hook di React che viene eseguito in risposta a cambiamenti specifici nello stato 
    dell'applicazione o al caricamento dei componenti nel DOM \cite{reactUseEffect}. Anche in questo caso tutti
    gli useEffect sono stati divisi in file in base alla loro semantica.

    \item \textbf{Centralizzazione degli \texttt{useState}}: uno \texttt{useState} è un hook di React che permette di aggiungere
    una variabile di stato locale a un componenete \cite{reactUseState}. Questi sono stati centralizzati utilizzando la Context API di React. 
    La pagina principale viene wrappata con un contesto, permettendo ai componenti figli di accedere e modificare lo stato globale 
    senza dover passare props attraverso vari livelli di componenti. 
\end{itemize}
\clearpage
\section{Backend}
Il backend del progetto è stato progettato seguendo principi che favoriscono la manutenibilità 
e l'efficienza delle operazioni di business. È organizzato secondo un'architettura a tre strati, che include Controller, 
Service e Repository.

\begin{itemize}

\item \textbf{Controller}: i Controller costituiscono il punto di ingresso delle richieste API provenienti dal frontend. 
Gestiscono la ricezione delle richieste HTTP, validazione dei dati e routing delle operazioni verso il Service appropriato. 
Questo livello fornisce un'interfaccia verso l'esterno dell'applicazione, separando la logica di gestione delle richieste 
dalla logica di business sottostante.

\item \textbf{Service}: il Service è responsabile della logica di business dell'applicazione. 
Ogni Service si occupa di una specifica area funzionale dell'applicazione, coordinando le operazioni necessarie per soddisfare 
le richieste ricevute dai Controller. Il Service coordina le interazioni con i Repository.

\item \textbf{Repository}: i Repository si interfacciano direttamente con le fonti di dati, che possono essere il database  
o API esterne di servizi di terze parti. Forniscono un'astrazione dell'accesso ai dati, permettendo al Service di accedere e manipolare 
i dati in modo uniforme, indipendentemente dalla specifica implementazione del sistema di persistenza dei dati. 
Questo approccio facilita la sostituzione o l'aggiornamento delle tecnologie di accesso ai dati senza modificare la 
logica di business.

\item \textbf{Gestione delle chiamate API esterne}: nel caso in cui le operazioni richiedano l'interazione con API esterne, 
i Repository sono progettati per gestire queste comunicazioni. Questo include la configurazione delle richieste HTTP, 
la gestione delle risposte e la gestione degli errori.

\end{itemize}