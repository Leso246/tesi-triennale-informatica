\subsection{Backend}
Di seguito viene riportato il codice del Controller riguardante l'eliminazione' dei meeting \ref{controller_eliminazione_meeting}: 
\subsubsection{Controller}
\begin{lstlisting}[language=java, frame=lines, basicstyle=\ttfamily\scriptsize, numbers=left, 
    caption={controller eliminazione meeting}, label={controller_eliminazione_meeting}]
@DeleteMapping(value = "/deleteMeet")
    public ResponseEntity<?> deleteMeeting(@RequestBody ObjectNode obj) {
		try {
			Mono<Boolean> resultMono = serviceMeet.deleteMeet(obj);
			Boolean result = resultMono.toFuture().get();

			if (result) {
				return new ResponseEntity<>(result, HttpStatus.OK);
			} else {
				return new ResponseEntity<>(result, HttpStatus.INTERNAL_SERVER_ERROR);
			}

		} catch (Exception e) {
			e.printStackTrace();
			return new ResponseEntity<>(e.getMessage(), HttpStatus.INTERNAL_SERVER_ERROR);
		}
}
\end{lstlisting}
\begin{itemize}
    \item \textbf{Riga 1}: l’annotazione indica che il metodo risponde a richieste HTTP
    DELETE sull’URL /deleteMeet.

    \item \textbf{Righe 4-5}: viene chiamato il service che restituisce un oggetto \texttt{Mono<Boolean>}. 
    L'esecuzione viene sospesa fino a quando il risultato non viene ottenuto tramite il metodo \texttt{toFuture().get()}.

    \item \textbf{Righe 7-15}: se l'operazione di eliminazione è riuscita, viene restituita una risposta con stato HTTP 200 (OK). 
    In caso contrario, viene restituita una risposta con stato HTTP 500 (Internal Server Error).
\end{itemize}
\subsubsection{Service}
Di seguito viene riportato il codice del Service riguardante l'eliminazione' dei meeting \ref{service_eliminazione_meeting}: 
\begin{lstlisting}[language=java, frame=lines, basicstyle=\ttfamily\scriptsize, numbers=left, 
    caption={service eliminazione meeting}, label={service_eliminazione_meeting}]
public Mono<Boolean> deleteMeet(ObjectNode obj) {
    try {
        Long idDatabase = obj.get("id").asLong();
        String webexId = obj.get("webexId").asText();

        return DeleteMeeting.deleteWebexMeet(webexAPI, webexId).flatMap(success -> {
            if (success) {
                DeleteMeeting.deleteDatabaseMeet(repo, idDatabase);
                return Mono.just(true);
            } else {
                return Mono.just(false);
            }
        }).doOnError(error -> {
            System.err.println("Errore durante la modifica del meeting su Webex: " + error.getMessage());
        }).onErrorReturn(false);

    } catch (Exception e) {
        e.printStackTrace();
        return Mono.just(false);
    }
}
\end{lstlisting}
\begin{itemize}
    \item \textbf{Righe 3-4}: vengono recupeati i campi \texttt{id} e \texttt{webexId} associati al meet passati dal frontend. 
    \item \textbf{Righe 6-8}: viene chiamato il metodo per eliminare il meet su Webex. Se l'operazione va a buon fine, 
    viene chiamato il metodo per eliminare il record del meeting a database. 
\end{itemize}
\subsubsection{Repository}
\noindent Metodo per eliminare il meet su Webex: viene effettuata una richiesta DELETE all'API \texttt{https://webexapis.com/v1/meetings/webexId} 
\ref{repository_eliminazione_meeting}.
L'unico parametro necessario è il \texttt{WebexId}, insieme al token per autenticare l'API.
\begin{lstlisting}[language=java, frame=lines, basicstyle=\ttfamily\scriptsize, numbers=left, 
    caption={chiamata all'API per eliminare un meet su Webex}, label={repository_eliminazione_meeting}]
 public static Mono<Boolean> deleteWebexMeet(WebClient webexAPI, String webexId) {
	String uri = MeetingAPI.DELETE_WEBEX_MEET + webexId;

	return webexAPI.delete()
	    .uri(uri)
	    .header(HttpHeaders.AUTHORIZATION, "Bearer " + MeetingAPI.TOKEN)
	    .exchangeToMono(response -> {
	        if (response.statusCode().is2xxSuccessful()) {
	            return Mono.just(true);
	        } else {
	            return response.createException().flatMap(Mono::error);
	        }
	    })
	    .onErrorResume(e -> Mono.just(false));
}
\end{lstlisting}
\noindent Metodo per eliminare il record del meet a database \ref{repository_eliminazione_meeting_database}: 
viene recuperata l'istanza del \texttt{Meet} grazie al suo ID
utilizzando il metodo \texttt{findMeetById}. Successivamente viene rimosso dal database.
\begin{lstlisting}[language=java, frame=lines, basicstyle=\ttfamily\scriptsize, numbers=left,
    caption={chiamata al metodo per eliminare un meet a database}, label={repository_eliminazione_meeting_database}]
public static void deleteDatabaseMeet(MeetJPA repo, Long id) {	
    try {     
        Meet meetDaEliminare = repo.findMeetById(id);      
        repo.delete(meetDaEliminare);      
    } catch (Exception e) {
        e.printStackTrace();
        throw new RuntimeException(
            "Errore durante l'eliminazione del meeting al database", e
        );
    }
}
\end{lstlisting}