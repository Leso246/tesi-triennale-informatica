\subsection{Frontend}
Attualmente, non è implementato alcun metodo per la creazione dei meeting attraverso il frontend del sistema. 
L'idea di base è che gli utenti possano cercare annunci di lavoro creati dalle aziende utilizzando una barra di ricerca dedicata.
Gli utenti potrebbero inserire parole chiave, filtri di posizione o settore, e altre specifiche per trovare gli annunci 
che corrispondono ai loro interessi e competenze. Allo stesso modo, le aziende potrebbero cercare utenti basandosi 
sulle informazioni presenti nei loro profili, come competenze, esperienze lavorative, certificazioni, e altre informazioni pertinenti. 
\\
\\
Dopo aver individuato un annuncio di interesse o un utente potenziale, è possibile richiedere un incontro 
mediante un apposito pulsante, attraverso il quale si forniscono le proprie disponibilità per il meeting. 
L'utente o l'azienda potrebbe specificare una serie di fasce orarie e date preferite, offrendo così flessibilità alla controparte. 
Nel caso in cui la controparte accetti l'invito scegliendo una data tra quelle disponibili, il sistema 
dovrebbe procedere automaticamente con la creazione del meeting, aggiornando i calendari delle parti coinvolte.
\\
\\
Tuttavia, sussiste un problema fondamentale: al momento, manca l'implementazione della logica di ricerca degli annunci 
e degli utenti, la creazione degli annunci da parte delle aziende e il completamento dei profili da parte degli utenti, 
rendendo impossibile anche la relativa visualizzazione. Di conseguenza, la funzionalità per la creazione automatica 
dei meeting non è stata implementata.