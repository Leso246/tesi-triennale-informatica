\subsubsection{CreateMeeting}
Metodo per creare il meet su Webex:
\begin{lstlisting}[language=java, frame=lines, basicstyle=\ttfamily\scriptsize, numbers=left]
public static Mono<JsonNode> addWebexMeet(
    WebClient webexAPI, 
    ObjectNode obj, 
    String usernameAzienda,
    String usernameUtente) {

  String requestBody = createAddWebexMeetBody(obj, usernameAzienda, usernameUtente);

  return webexAPI.post()
    .uri(MeetingAPI.ADD_WEBEX_MEET)
    .header(HttpHeaders.AUTHORIZATION, "Bearer " + MeetingAPI.TOKEN)
    .contentType(MediaType.APPLICATION_JSON)
    .bodyValue(requestBody)
    .retrieve()
    .bodyToMono(String.class).map(responseBody -> {

        ObjectMapper objectMapper = new ObjectMapper();
            
        try {
            JsonNode jsonResponse = objectMapper.readTree(responseBody);
            return jsonResponse;
        } catch (JsonProcessingException e) {
            throw new RuntimeException("Errore durante il parsing della risposta JSON", e);
        }
    }).onErrorMap(e -> 
                    new RuntimeException("Errore durante la chiamata all'API di Webex", e));
}
\end{lstlisting}
\begin{itemize}
    \item \textbf{Riga 7}: \texttt{createAddWebexMeetBody} è un metodo di utilità che permette di creare il body
    della richiesta per l'API di Webex. Esempio:
    \begin{lstlisting}[language=json,firstnumber=1]
{
    "title": "Riunione tra Mario e Tech SPA",
    "start": "2024-06-25T16:00:00",
    "end": "2024-06-25T17:00:00",
    "invitees": [
        {
            "email": "techSPA@gmail.com",
            "displayName": "Tech SPA",
            "coHost": false
        },
        {
            "email": "mario@gmail.com",
            "displayName": "Mario",
            "coHost": false
        }
    ],
    "timezone": "Europe/Rome",
    "enabledJoinBeforeHost": true,
    "enableConnectAudioBeforeHost": true,
    "joinBeforeHostMinutes": 15,
    "scheduledType": "meeting",
    "enableAutomaticLock": true,
    "automaticLockMinutes": 60
}
    \end{lstlisting}
    \begin{itemize}
        \item \textbf{title}: titolo della riunione.
        \item \textbf{start}: data e ora d'inizio della riunione, formattata secondo lo standard ISO 8601.
        \item \textbf{end}: data e ora di fine della riunionw, formattata secondo lo standard ISO 8601.
        \item \textbf{invitees}: lista degli invitati alla riunione. Webex invierà una mail agli indirizzi specificati, il nome
        visualizzato è il valore contenuto in \texttt{displayName}.
        \item \textbf{timezone}: il fuso orario in cui si terrà il meet.
        \item \textbf{enabledJoinBeforeHost}: permette ai partecipanti di unirsi alla riunione anche senza la presenza dell'host. 
        Questo attributo è fondamentale poiché l'azienda è l'organizzatrice di tutti i meeting ma non parteciperà a nessuno di essi.
        \item \textbf{enableConnectAudioBeforeHost}: permette ai partecipanti di utilizzare l'audio anche in assenza dell'host.
        \item \textbf{joinBeforeHostMinutes}: indica quanti minuti rispetto alla data di inizio della riunione i partecipanti possono unirsi anche senza la presenza dell'host.
        \item \textbf{scheduledType}: indica il tipo di riunione, in questo caso è un meeting.
        \item \textbf{enableAutomaticLock}: dopo un certo periodo di tempo, nessuno può partecipare al meeting senza essere accettato dall'host.
        \item \textbf{automaticLockMinutes}: indica il periodo di tempo dopo il quale la riunione verrà automaticamente bloccata. 
        Questo valore è calcolato come la differenza tra \texttt{end} e \texttt{start}, in modo che, dopo la data di fine prefissata, 
        nessuno possa più partecipare.
    \end{itemize}

    \item \textbf{Righe 9-15}: viene effettuata una chiamata POST all'API 
    \\ \texttt{https://webexapis.com/v1/meetings}, utilizzando il token d'autorizzazione \hyperref[sec:recupero_access_token]{generato} 
    e passando come body il JSON precedentemente creato.

    \item \textbf{Righe 20-21}: se la richiesta è andata a buon fine, viene restituito come risposta il JSON fornito dall'API di Webex.
    
\end{itemize}
Metodo per creare il meet a database:
\begin{lstlisting}[language=java, frame=lines, basicstyle=\ttfamily\scriptsize, numbers=left]
public static void addMeetToDatabase(
    ObjectNode obj, 
    Azienda azienda, 
    Utente utente, 
    MeetJPA repo, 
    String link,
    String webexId) {

  try {
    
    Meet meet = new Meet();
  
    Timestamp dataInizio = Timestamp.valueOf(UtiltyMeet.transformUTCIntoRomeTimezone(obj.get("data_inizio").asText()));
    Timestamp dataFine = Timestamp.valueOf(UtiltyMeet.transformUTCIntoRomeTimezone(obj.get("data_fine").asText()));

    meet.setDataInizio(dataInizio);
    meet.setDataFine(dataFine);
    meet.setAzienda(azienda);
    meet.setUtente(utente);
    meet.setLink(link);
    meet.setWebexId(webexId);

    repo.save(meet);

  } catch (Exception e) {
    throw new RuntimeException("Errore durante la creazione del meeting al database", e);
  }
}
\end{lstlisting}
Questo metodo utilizza JPA per creare facilmente un record nella tabella \textbf{meet} del database.
\\
\\
\textit{Nota: il frontend fornisce le date nel fuso orario UTC+0. Pertanto, è stato creato 
un metodo \texttt{transformUTCIntoRomeTimezone} per convertire le date nel fuso orario di Roma. 
Questo garantisce che le date siano corrette rispetto alla localizzazione del sistema.}