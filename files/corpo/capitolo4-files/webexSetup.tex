\section{Configurazione di Webex}
Per utilizzare i servizi offerti da Cisco Webex, è necessario compiere una serie di passaggi
preliminari per ottenere un \textbf{access token}, che verrà utilizzato per autenticare tutte le chiamate alle API.

\begin{enumerate}

    \item \textbf{Creazione di un account}: il primo passo consiste nel creare
    un account sul sito \href{https://developer.webex.com/}{Webex for Developers}, che permette 
    l'accesso agli strumenti necessari per lo sviluppo.
    
    \item \textbf{Creazione dell'integrazione}: successivamente si deve creare l'integrazione per 
    l'applicazione sul proprio account. Durante questo processo sono stati specificati:
        \begin{itemize}
            \item Il \textbf{nome dell'applicazione}
            \item Una \textbf{descrizione dell'applicazione}
            \item Un'\textbf{icona rappresentativa}
            \item Gli \textbf{scopes}: di fondamentale importanza sono gli scopes. Questi definiscono tutte le operazioni
            che il nostro account potrà andare ad eseguire  quando si effettuano chiamate alle API. È necessario
            selezionare tutti gli scopes necessari alle finalità del progetto, con la possibilità di abilitarli 
            o disabilitarli in qualsiasi momento.
        \end{itemize}
 
    \item \textbf{Richiesta di un account sandbox}: per effettuare test senza limitazioni durante lo sviluppo,
    è necessario richiedere un account Sandbox tramite il proprio account creato in precedenza. Una volta ottenute
    le credenziali, si può usufruire di questo account speciale per accedere come amministratore di un'organizzazione 
    e gestire tutti i meeting e le relative impostazioni attraverso una comoda interfaccia, 
    \href{https://admin.webex.com/login}{Webex Control Hub}.
    \cite{WebexSandboxAccount}
    
    \item \textbf{Recupero dell'access token}: l'ultimo passaggio è stato ottenere l'access token 
    necessario per autenticare le chiamate alle API.
    Bisogna effettuare una richiesta POST all'API
    \texttt{https://webexapis.com/v1/access\_token} con il seguente body:
    \begin{lstlisting}[language=json,firstnumber=1]
{
    "grant_type": "authorization_code",
    "client_id": "1234567890abcdef123456",
    "client_secret": "abcdef1234567890abcdef1234567890",
    "code":"12345678abcdef12345678abcdef"
}
    \end{lstlisting}

    I valori di \texttt{client\_id}, \texttt{client\_secret}, e \texttt{code} sono stati reperiti dalla pagina di integrazione 
    dell'applicazione sul proprio account Webex for Developers.
    Se tutti i dati passati sono corretti, l'access token viene restituito come risposta.
 
\end{enumerate}
\label{sec:recupero_access_token}