\subsection{Frontend}
Ogni volta che la pagina viene caricata, viene chiamato uno \texttt{useEffect} \ref{useEffect_chiamato_al_caricamento_della_pagina}.
\begin{lstlisting}[language=typescript, frame=lines, basicstyle=\ttfamily\scriptsize, numbers=left,
    caption={useEffect chiamato al caricamento della pagina}, label={useEffect_chiamato_al_caricamento_della_pagina}]
useEffect(() => {
    const jwtToken = TokenAuth.getTokenAuth();
    
    const getMeet = async () => {
        try {
           const response = await ritornaMeeting(jwtToken);
           setEvents(response.data);
        } catch (error) {
           showPopup('Errore durante il caricamento dei colloqui', false);
           console.error('Errore durante il recupero degli eventi:', error);
        }
    };  
    getMeet();
}, []);
\end{lstlisting}
\begin{itemize}
    \item \textbf{Riga 2}: viene recuperato il \texttt{jwtToken} dell'utente. Abbreviazione di \textbf{JSON Web Token}, 
    il \texttt{jwtToken} è uno standard (RFC 7519) per trasmettere in modo sicuro dei dati sotto forma di oggeto JSON. \cite{jwtToken}
    Viene generato quando l'utente effettua il login e contiene informazioni utili riguardo a esso.
    \item \textbf{Righe 4-11}: viene definita la funzione \texttt{getMeet} che chiama il backend per il recupero degli eventi \ref{api_recupero_eventi_frontend}
        \begin{itemize}
            \item \textbf{API}:
            \begin{lstlisting}[language=typescript, frame=lines, basicstyle=\ttfamily\scriptsize, numbers=left, 
                label={api_recupero_eventi_frontend}, caption={API definita a frontend per il recupero degli eventi}]
const api_url =
  import.meta.env.MODE === 'development'
    ? import.meta.env.VITE_API_GESTUT_URL_DEV
    : import.meta.env.VITE_API_GESTUT_URL_PROD;

const get_meet = api_url + 'meet/getMeet?jwtToken=';

export const ritornaMeeting = (jwtToken: string) => {
  return axios.get(get_meet + jwtToken);
};

            \end{lstlisting}
        \end{itemize}

    \item \textbf{Riga 7}: \texttt{setEvents} è uno \texttt{useState}.  In questo caso è usato per impostare gli eventi 
    di \texttt{Fullcalendar} in modo da poterli visualizzare. \texttt{Fullcalendar} accetta un array di eventi così formattati \ref{formattazione_eventi_fullcalendar}:
\begin{lstlisting}[language=json, label={formattazione_eventi_fullcalendar}, caption={formattazione eventi FullCalendar}]
[
    {
        "id": 50,
        "start": "2024-06-28T15:00:00",
        "end": "2024-06-28T16:00:00",
        "editable": true|false,
        "eventStartEditable": true|false,
        "extendedProps": {
            "nomeUtente": "Mario",
            "nomeAzienda": "Tech SPA",
            "link": "webex.com/meet/ex1",
            "webexId": "93d7d864bd9b4"
        }
    },                   
]               
\end{lstlisting}
    \begin{itemize}
        \item \textbf{Riga 6} e \textbf{Riga 7}: sono attributi degli eventi di \texttt{Fullcalendar}
        che permettono di trascinare un evento sul calendario per modificarne la data e/o l'ora. Questo comportamento verrà
        discusso nella sezione dedicata alla modifica dei meeting.
        \item \textbf{Righe 8-13}: in questo JSON è possibile inserire tutte le coppie di chiave-valore che risultano
        utili allo sviluppo. In questo caso, di particolare importanza sono il \texttt{link} del meeting e il suo \texttt{webexId}.
    \end{itemize}

    \item \textbf{Righe 9-10}: nel caso in cui il caricamento dei colloqui fallisca,
    utilizzando nuovamente degli \texttt{useState} viene mostrato un popup all'utente 
    contenente il messaggio di errore \textit{"Errore durante il caricamento dei colloqui"}.
\end{itemize}
Una volta che gli eventi sono stati caricati correttamente, vengono visualizzati come 
\hyperref[fig:visualizzazioneCalendario]{precedentemente mostrato} all'interno del calendario.
\\
\\
Quando si clicca su un evento, viene visualizzato un \texttt{Dialog} contenente i dettagli 
del meet recuperati dalla funzione \texttt{getMeet}.
Questo \texttt{Dialog} cambia a seconda che l'utente loggato sia un candidato \ref{visualizzazione_candidato_evento_non_iniziato} 
o un'azienda \ref{visualizzazione_azienda_evento_non_iniziato}, sfruttando la renderizzazione condizionale di React. 
\cite{reactConditionalRendering}
\begin{figure}[H]
    \centering
    \includegraphics[width=\textwidth]{DettagliMeet/DettagliMeetUtente.png}
    \caption{Visualizzazione candidato - evento non iniziato}
    \label{visualizzazione_candidato_evento_non_iniziato}
\end{figure}
\begin{figure}[H]
    \centering
    \includegraphics[width=\textwidth]{DettagliMeet/DettagliMeetAziendaDark.png}
    \caption{Visualizzazione azienda - evento non iniziato}
    \label{visualizzazione_azienda_evento_non_iniziato}
\end{figure}
\noindent Quando mancano meno di 10 minuti all'inizio del meet, il pulsante per partecipare 
al meet viene abilitato, come mostrato in figura \ref{visualizzazione_azienda_evento_iniziato}. 
Cliccandoci, si viene reinderizzati al meet su Webex.
In questo stato, l'evento non è più modificabile né eliminabile.
\begin{figure}[H]
    \centering
    \includegraphics[width=\textwidth]{DettagliMeet/VaiAlMeet.png}
    \caption{Visualizzazione azienda - evento iniziato}
    \label{visualizzazione_azienda_evento_iniziato}
\end{figure}
\noindent Se l'evento è già concluso, è comunque possibile visualizzarne i dettagli 
per avere uno storico dell'incontro, come mostrato in figura \ref{visualizzazione_azienda_evento_concluso}. 
In questo caso, non è possibile effettuare nessun'altra operazione sul meet.
\begin{figure}[H]
    \centering
    \includegraphics[width=\textwidth]{DettagliMeet/EventoConcluso.png}
    \caption{Visualizzazione azienda - evento concluso}
    \label{visualizzazione_azienda_evento_concluso}
\end{figure}
\noindent La logica per abilitare/disabilitare i pulsanti è gestita interamente a frontend 
attraverso l'utilizzo di uno \texttt{useEffect} \ref{useEffect_per_la_logica_dei_pulsanti}: ogni volta che un evento viene selezionato
e viene aperto il relativo \texttt{Dialog}, il sistema verifica la data attuale 
rispetto alla data di inizio e di fine del meeting. Questo controllo determina se mancano più di 10 
minuti all'inizio del meeting, se il meeting è già iniziato o se è concluso. Sulla base di questi valori, 
i pulsanti vengono abilitati o disabilitati di conseguenza.
\\ 
\\
È importante notare che durante la creazione del meeting, l'orario di accesso degli utenti è impostato 15 minuti 
prima della data di inizio effettiva. Questa impostazione consente agli utenti di accedere direttamente al meeting 
senza attese o conferme, in quanto il pulsante "Vai al meet" viene abilitato 10 minuti prima dell'inizio programmato.
\begin{lstlisting}[language=typescript, frame=lines, basicstyle=\ttfamily\scriptsize, numbers=left, 
    label={useEffect_per_la_logica_dei_pulsanti}, caption={useEffect per la logica dei pulsanti}]
useEffect(() => {
    if (!selectedEvent) return;
      
    const { start, end } = selectedEvent;
    const now = new Date();
      
    const eventStartDate = new Date(start);
    const eventFinishDate = new Date(end);
      
    const minutesUntilStart = differenceInMinutes(eventStartDate, now);
    const isLessThan10Minutes = minutesUntilStart < 10;
    const isEventFinished = now > eventFinishDate;
      
    setIsLessThan10Minutes(isLessThan10Minutes);
    setIsEventFinished(isEventFinished);
}, [selectedEvent]);
\end{lstlisting}
\subsubsection{Recupero degli invitati}
Per quanto riguarda gli invitati a un meet, questi sono visualizzabili \textbf{solo} dall'azienda. 
Poiché non vengono salvati nel database, è necessario fare una chiamata all'API di Webex per recuperarli.
Considerato troppo oneroso effettuare una chiamata per ogni singolo meet durante il caricamento della 
pagina, l'API \texttt{getMeetInvitees} \ref{getMeetInvitees} viene chiamata ogni 
volta che si clicca su un evento. Sono stati effettuati 30 test e il tempo di risposta medio è 
risultato essere di 863 ms, motivo per cui questa soluzione è stata ritenuta adeguata. 
Prima che gli invitati vengano caricati, viene visualizzata 
una semplice scritta \textit{"Caricamento..."} di fianco alla dicitura \textit{"Lista degli invitati:"}
del \texttt{Dialog}.

\begin{lstlisting}[language=typescript, frame=lines, basicstyle=\ttfamily\scriptsize, numbers=left, 
    label={getMeetInvitees}, caption={chiamata all'api getMeetInvitees effettuata se l'utente è un'azienda}]
const handleEventClick = async (arg: any) => {
    setInvitees([]); // Resetta l'array di invitati
    setSelectedEvent(arg.event);
    setOpen(true);
    if (isAzienda) {
        const payload = {
        id: arg.event.id,
        webexId: arg.event.extendedProps.webexId,
        };   
        try {
            const response = await getMeetInvitees(payload);
            if (response.status === 200) {
              setInvitees(response.data);
            } else {
              console.error('Errore nel recupero degli invitati:', response.statusText);
            }
          } catch (error) {
            console.error('Errore nel recupero degli invitati:', error);
          }
        } ...
\end{lstlisting}