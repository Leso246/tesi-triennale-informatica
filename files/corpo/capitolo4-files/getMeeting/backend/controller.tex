\subsubsection{Controller}
Di seguito viene riportato il codice del Controller riguardante il recupero dei meeting \ref{controller_recupero_meeting}: 
\begin{lstlisting}[language=java, frame=lines, basicstyle=\ttfamily\scriptsize, numbers=left, 
  caption={controller recupero meeting}, label={controller_recupero_meeting}]
@GetMapping(value = "/getMeet")
public ResponseEntity<?> getMeeting(@RequestParam("jwtToken") String jwtToken) {
  try {
     String email = gestioneToken.getEmailFromToken(jwtToken);
     boolean isAzienda = gestioneToken.isAzienda(jwtToken);

	  Long idUtente = 0l;

	  if (isAzienda) {
		  idUtente = serviceAzienda.trovaPerEmail(email).getId();
	  } else {
		  idUtente = serviceUtente.trovaPerEmail(email).getId();
	  }

	  List<Map<String, Object>> meets = serviceMeet.getMeet(isAzienda, idUtente);

	  return new ResponseEntity<>(meets, HttpStatus.OK);

   } catch(Exception e) {
     e.printStackTrace();
     return new ResponseEntity<>(e.getMessage(), HttpStatus.INTERNAL_SERVER_ERROR);
   }
}    
\end{lstlisting}
\begin{itemize}
    \item \textbf{Riga 1}:
    l'annotazione indica che il metodo risponde a richieste HTTP GET sull'URL /getMeet.
  
    \item \textbf{Riga 2}:
    questo è il metodo che gestisce la richiesta HTTP. 
        \begin{itemize}         
            \item \textbf{@RequestParam("jwtToken") String jwtToken}: è un'annotazione di Spring che indica 
            che la richiesta deve contenere un parametro \texttt{jwtToken}, 
            il quale assegnato alla variabile di tipo \texttt{String} denominata \texttt{jwtToken}.      
        \end{itemize}
  
    \item \textbf{Righe 4-13}:
    viene recuperata l'email dal token e verifica se il token appartiene a un'azienda o a un utente. 
    Una volta determinato il tipo di utente, viene recuperato l'ID corrispondente nella rispettiva tabella. 
    Questo passaggio è cruciale perché è necessario mostrare solo i meet specifici di quell'utente e l'ID, 
    che non è salvato nel \texttt{jwtToken}, è utilizzato come riferimento.
  
    \item \textbf{Riga 15}:
    viene chiamato il service per recuperare i meet da database.
  
    \item \textbf{Righe 19-21}:
    se si verifica un'eccezione durante l'esecuzione, l'eccezione viene stampata nello stack trace e viene 
    restituita una risposta con lo stato INTERNAL_SERVER_ERROR (500), contenente il messaggio dell'eccezione.
  \end{itemize}