\subsubsection{Service}
Di seguito viene riportato il codice del Service riguardante il recupero dei meeting \ref{service_recupero_meeting}: 
\begin{lstlisting}[language=java, frame=lines, basicstyle=\ttfamily\scriptsize, numbers=left, 
  label={service_recupero_meeting}, caption={service recupero meeting}]
@Autowired
private MeetJPA repo;

public  List<Map<String, Object>> getMeet(boolean isAzienda, Long idUtente) {
  List<MeetDTO> meets;

	if (isAzienda) {
		meets =  repo.getMeetAzienda(idUtente);
	} else {
		meets = repo.getMeetUtente(idUtente);
	}

  List<Map<String, Object>> transformedMeets = 
      meets.stream()
		  .map(meet -> UtiltyMeet.transformMeetDTO(meet, isAzienda))
        .collect(Collectors.toList());

  return transformedMeets;
}
\end{lstlisting}
\begin{itemize}
    \item \textbf{Righe 1-2}: questa annotazione indica a Spring di iniettare automaticamente un'istanza del bean \texttt{MeetJPA}
    nella variabile \texttt{repo} quando viene creata un'istanza di \texttt{MeetService}. \cite{AutoWiredSpring}
    In questo contesto, la classe \texttt{MeetService} ha bisogno di accedere al repository \texttt{MeetJPA} per effettuare
    operazioni CRUD sui meet.

    \item \textbf{Righe 8-12}: a seconda che l'utente sia un'azienda o un candidato, vengono recuperati i meet associati.
    
    \item \textbf{Righe 14-17}: metodo di utilità che itera sull'array di JSON restituito dal
    database e formatta ogni elemento in modo che possa essere letto correttamente da \texttt{FullCalendar}.
\end{itemize}