\subsubsection{Repository}
Per modificare l'orario di un meeting, viene effettuata una richiesta PATCH 
all'API \texttt{https://webexapis.com/v1/meetings/webexId}.
\begin{lstlisting}[language=java, frame=lines, basicstyle=\ttfamily\scriptsize, numbers=left]
public static Mono<Boolean> modifyWebexMeetDate(WebClient webexAPI, JsonNode evento) {
	try {
		ObjectNode requestBody = UtiltyMeet.createModifyWebexMeetDateBody(evento);
		String webexId = evento.get("extendedProps").get("webexId").asText();

		String uri = MeetingAPI.MODIFY_WEBEX_MEET_DATE + webexId;

		return webexAPI
			.patch()
			.uri(uri)
			.header(HttpHeaders.AUTHORIZATION, "Bearer " + MeetingAPI.TOKEN)
			.contentType(MediaType.APPLICATION_JSON)
			.bodyValue(requestBody)
			.retrieve()
			.toBodilessEntity()
			.thenReturn(true).onErrorMap(e -> new RuntimeException(
				"Errore durante la chiamata all'API di Webex per la modifica di un evento", e));

	} catch (Exception e) {
		e.printStackTrace();
		return Mono.just(false);
	}
}
\end{lstlisting}
\begin{itemize}
    \item \textbf{Riga 3}: metodo di utilità per costruire il body della richiesta
        \begin{lstlisting}[language=json,firstnumber=1]
{
    "start": "2024-07-22T10:00:00",
    "end": "2024-07-22T11:00:00",
    "timezone": "Europe/Rome"
}
        \end{lstlisting}
    
    \item \textbf{Righe 8-17}: chiamata all'API.
\end{itemize}
Per aggiungere gli invitati a un meeting, viene effettuata una richiesta POST 
all'API \texttt{https://webexapis.com/v1/meetingInvitees/bulkInsert}.
\begin{lstlisting}[language=java, frame=lines, basicstyle=\ttfamily\scriptsize, numbers=left]
public static Mono<Boolean> modifyWebexMeetInvitees(WebClient webexAPI, JsonNode evento, List<String> invitatiList) {
	try {		
		if (invitatiList.size() != 0) {		
			ObjectNode requestBody =  createModifyWebexInviteesBody(evento, invitatiList);

			return webexAPI.post().uri(MeetingAPI.MODIFY_WEBEX_MEET_INVITEES)
				.header(HttpHeaders.AUTHORIZATION, "Bearer " + MeetingAPI.TOKEN)
				.contentType(MediaType.APPLICATION_JSON).bodyValue(requestBody).retrieve().toBodilessEntity()
				.thenReturn(true).onErrorResume(e -> {
					System.err.println(
						"Errore durante la chiamata all'API di Webex per l'aggiunta degli invitati a un meet: "
						+ e.getMessage());

				    return Mono.error(e);
				});
		} else {
			return Mono.just(true);
		}
	} catch (Exception e) {
		e.printStackTrace();
		return Mono.error(e);
	}
}
\end{lstlisting}
\begin{itemize}
    \item \textbf{Riga 3}: la richiesta viene eseguita solo se ci sono invitati da aggiungere. In caso contrario, viene restituito 
    \texttt{true} e il controllo torna al service.

    \item \textbf{Riga 4}: metodo di utilità per costruire il body della richiesta. 
    \texttt{hostEmail} rappresenta l'email dell'account che crea tutti i meeting, 
    mentre \texttt{items} è un array di JSON contenente le email degli invitati. È possibile specificare ulteriori attributi
     per ogni invitato, ma in questo caso non è stato necessario.
    \begin{lstlisting}[language=json,firstnumber=1]
{
    "meetingId":"abcde123456",
    "hostEmail":"admin@tiscali.it",
    "items":[
        {
            "email":"test@gmail.com"
        },
        {
            "email":"popi@gmail.com"
        }
    ]
}
    \end{lstlisting}

    \item \textbf{Righe 6-15}: viene chiamata l'API.
\end{itemize}
La modifica del meeting nel database prevede l'aggiornamento delle date di inizio e fine, 
poiché gli invitati non vengono salvati nel database ma vengono recuperati direttamente da Webex ogni volta.
\begin{lstlisting}[language=java, frame=lines, basicstyle=\ttfamily\scriptsize, numbers=left]
    public static void modifyDatabaseMeet(MeetJPA repo, JsonNode evento) {
		try {
			Meet meet = repo.findMeetById(evento.get("id").asLong());

			Timestamp dataInzio = 
                Timestamp.valueOf(UtiltyMeet.transformUTCIntoRomeTimezone(evento.get("start").asText()));
			Timestamp dataFine = 
                Timestamp.valueOf(UtiltyMeet.transformUTCIntoRomeTimezone(evento.get("end").asText()));

			meet.setDataInizio(dataInzio);
			meet.setDataFine(dataFine);

			repo.save(meet);

		} catch (Exception e) {
			e.printStackTrace();
			throw new RuntimeException("Errore durante la modifica del meeting a database", e);
		}
	}
\end{lstlisting}
\begin{itemize}
    \item \textbf{Riga 3}: ricerca dell'oggetto \texttt{Meet} nel database utilizzando il suo ID.
    \item \textbf{Righe 5-8}: calcolo dei \texttt{Timestamp} nel fuso orario di Roma.
    \item \textbf{Righe 10-11}: aggiornamento delle date di inizio e fine dell'oggetto \texttt{Meet}.
    \item \textbf{Riga 13}: salvataggio della modifica nel database.
\end{itemize}