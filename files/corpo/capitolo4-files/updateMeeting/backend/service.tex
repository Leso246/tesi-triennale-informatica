\subsubsection{Service}
\begin{lstlisting}[language=java, frame=lines, basicstyle=\ttfamily\scriptsize, numbers=left]
public Mono<Boolean> modifyMeet(ObjectNode obj) {
    try {
        JsonNode evento = obj.get("evento");
        
        JsonNode invitati = obj.get("invitati");    
        List<String> invitatiList = new ArrayList<>();
        if (invitati != null && invitati.isArray()) {
            for (JsonNode invitato : invitati) {
                invitatiList.add(invitato.asText());
            }
        }

        return modifyWebexMeet(webexAPI, evento, invitatiList).flatMap(success -> {
            if (success) {
                ModifyMeeting.modifyDatabaseMeet(repo, evento);
                return Mono.just(true);
            } else {
                return Mono.just(false);
            }
        }).doOnError(error -> {
            System.err.println(
                "Errore durante la modifica del meeting su Webex: " + error.getMessage());
        }).onErrorReturn(false);

    } catch (Exception e) {
        e.printStackTrace();
        return Mono.just(false);
    }
}
\end{lstlisting}
\begin{itemize}
    \item \textbf{Righe 5-11}: viene estratto dal body della richiesta l'oggetto \textit{invitati}, 
    contenente le email degli invitati da aggiungere al meeting. Questo viene successivamente convertito in un array di stringhe.
    
    \item \textbf{Riga 13}:  viene invocato il metodo \texttt{modifyWebexMeet}, che si trova anch'esso nel Service. 
    Questo metodo gestisce la logica necessaria per effettuare le due chiamate API a Webex: una per aggiornare 
    l'orario e l'altra per aggiungere gli invitati.

    \item \textbf{Riga 15}: se entrambe le chiamate a Webex hanno esito positivo, 
    viene chiamato il metodo per aggiornare l'orario del meeting nel database.
\end{itemize}