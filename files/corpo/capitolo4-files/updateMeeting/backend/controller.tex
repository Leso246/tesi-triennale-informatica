\subsubsection{Controller}
Di seguito viene riportato il codice del Controller riguardante la modifica dei meeting \ref{controller_modifica_meeting}: 
\begin{lstlisting}[language=java, frame=lines, basicstyle=\ttfamily\scriptsize, numbers=left,
    caption={controller modifica meeting}, label={controller_modifica_meeting}]
@PostMapping(value = "/modifyMeet", consumes = "application/json")
public ResponseEntity<?> modifyMeeting(@RequestBody ObjectNode obj) {
    try {   
        Mono<Boolean> resultMono = serviceMeet.modifyMeet(obj); 
        Boolean result = resultMono.toFuture().get();
    
        if (result) {
            return new ResponseEntity<>(result, HttpStatus.OK);
        } else {
             return new ResponseEntity<>(result, HttpStatus.INTERNAL_SERVER_ERROR);
        }  
    } catch (Exception e) {
        e.printStackTrace();
        return new ResponseEntity<>(e.getMessage(), HttpStatus.INTERNAL_SERVER_ERROR);
    }
}
\end{lstlisting}
\begin{itemize}
    \item \textbf{Riga 1}:
    l'annotazione indica che il metodo risponde a richieste HTTP POST sull'URL /addMeet 
    e che il body della richiesta deve essere in formato JSON.
  
    \item \textbf{Riga 2}:
    questo è il metodo che gestisce la richiesta HTTP. 
        \begin{itemize}
            
            \item \textbf{@RequestBody ObjectNode obj}: è un'annotazione di Spring che indica al framework di 
            utilizzare il corpo della richiesta HTTP per popolare \texttt{obj}.
            
        \end{itemize}
  
    \item \textbf{Riga 4}:
    all'interno del blocco try, viene chiamato il \textbf{service} attraverso
    \texttt{serviceMeet.modifyMeet(obj)}, che restituisce un oggetto 
    \texttt{Mono\textless{}Boolean\textgreater{}}. 
  
    \item \textbf{Riga 5}:
    il risultato del Mono viene convertito in un Future e poi ottenuto. Questo metodo blocca l'esecuzione 
    finché il risultato non è disponibile.
\end{itemize}