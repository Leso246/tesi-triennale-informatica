\subsection{Frontend}
La modifica dei meeting è disponibile unicamente per le aziende. È possibile:
\begin{itemize}
    \item Modificare data e ora del colloquio.
    \item Aggiungere ulteriori invitati, inserendo le rispettive email.
\end{itemize}
Per poter accedere alla sezione di modifica di un evento, è necessario aprirne i dettagli e cliccare su \textit{"Modifica"}. 
Il \texttt{Dialog} \ref{visualizzazione_modifica} caricato è lo stesso utilizzato per visualizzare i dettagli, sfruttando ancora una volta la renderizzazione
condizionale di React . \cite{reactConditionalRendering}
\begin{figure}[H]   
    \centering
    \includegraphics[width=\textwidth]{ModificaMeet/visualizzazioneModifica.png}
    \caption{Visualizzazione Modifica}
    \label{visualizzazione_modifica}
\end{figure}
\subsubsection{Modifica di data e ora}
Per cambiare la data e l'ora di un meeting, sono disponibili due modalità: 
\begin{itemize}
    % Sezione dedicata alla scelta dal DateTimePicler
    \item selezionare la data e l'ora dai rispettivi \texttt{DateTimePicker} \ref{modifica_di_data_e_ora}. 
    Tutte le date passate sono disabilitate e quindi non selezionabili.
    \begin{figure}[H]   
        \centering
        \includegraphics[width=\textwidth]{ModificaMeet/modificaData.png}
        \caption{Modifica di data e ora}
        \label{modifica_di_data_e_ora}
    \end{figure}
    Ci sono tre condizioni da rispettare, le ultime due imposte da Webex: 
        \begin{itemize}
            \item la data di fine non può essere antecedente alla data di inizio \ref{errore_data_fine}.
            \item un meet non può avere una durata inferiore ai 10 minuti \ref{errore_10_minuti}.
            \item un meet non può avere una durata superiore alle 24 ore \ref{errore_24_ore}.
        \end{itemize}
    Nel caso si verifichi una di queste tre condizioni, viene visualizzato un messaggio di errore e il pulsante per salvare viene disabilitato.
    \begin{figure}[H]
        \centering
        \begin{minipage}{0.45\textwidth}
            \centering
            \includegraphics[width=\textwidth]{ModificaMeet/erroreDataFine.png}
            \caption{errore data fine}
            \label{errore_data_fine}
        \end{minipage}
        \hspace{0.05\textwidth}
        \begin{minipage}{0.45\textwidth}
            \centering
            \includegraphics[width=\textwidth]{ModificaMeet/errore10min.png}
            \caption{errore 10 minuti}
            \label{errore_10_minuti}
        \end{minipage}
        \hspace{0.05\textwidth}
        \begin{minipage}{0.45\textwidth}
            \centering
            \includegraphics[width=\textwidth]{ModificaMeet/errore24h.png}
            \caption{errore 24 ore}
            \label{errore_24_ore}
        \end{minipage}
    \end{figure}
    % Sezione dedicata al drag degli eventi
    \item Trascinando gli eventi a calendario \ref{trascinamento_evento}, rilasciandoli sulla data e l'ora desiderate:
        \begin{itemize}
            \item Nella visualizzazione mensile, è possibile cambiare solo la data e non l'ora.
            \item Nella visualizzazione settimanale, è possibile cambiare la data e l'ora con una precisione di mezz'ora rispetto
            all'orario di inizio del meeting.
            \item Nella visualizzazione giornaliera è possibile cambiare unicamente l'ora, sempre con una precisione di mezz'ora.
        \end{itemize}
    Se la modifica ha avuto successo, viene visualizzato un popup per avvisare l'utente \ref{successo_nella_modifica}.
        \begin{figure}[H]
            \centering
            \begin{minipage}{0.45\textwidth}
                \centering
                \includegraphics[width=\textwidth]{ModificaMeet/spostamentoEvento.PNG}
                \caption{trascinamento evento}
                \label{trascinamento_evento}
            \end{minipage}
            \hspace{0.05\textwidth}
            \begin{minipage}{0.45\textwidth}
                \centering
                \includegraphics[width=\textwidth]{ModificaMeet/successoModifica.PNG}
                \caption{successo nella modifica}
                \label{successo_nella_modifica}
            \end{minipage}
        \end{figure}
        Nel caso in cui si cerchi di spostare un evento in una data passata, questo viene riportato alla sua posizione originale 
        e l'utente viene avvisato dell'errore tramite un popup \ref{data_passata_selezionata}.
    \begin{figure}[H]   
        \centering
        \includegraphics[width=\textwidth]{ModificaMeet/dataInizioPassata.png}
        \caption{data passata selezionata}
        \label{data_passata_selezionata}
    \end{figure}
    Anche quando viene riscontrato un errore nella modifica, l'utente viene avvisato attraverso un popup e l'evento riportato
    alla posizione originale.
\end{itemize}
\subsubsection{Aggiunta degli invitati}
Per aggiungere invitati a un meet tramite email, è sufficiente digitare gli indirizzi all'interno dell'apposito \texttt{TextField}. 
Quando l'utente inserisce un'email e sposta il focus su un altro elemento, oppure digita uno tra i seguenti caratteri:
\begin{center}
\begin{tabular}{c|c|c|c|c}
Spazio & Invio & , & ; & . \
\end{tabular}
\end{center}
viene effettuato un controllo sull'input digitato. Se l'input viene riconosciuto, attraverso una regex, come un'email valida, 
questa viene visualizzata all'interno di una card per indicare che è stata correttamente riconosciuta e viene contemporaneamente
aggiunta all'array \textit{inviteesList}, uno \texttt{useState}, che verrà inviato al backend. Viene riportato un esempio
dell'interfaccia in figura \ref{aggiunta_degli_invitati}.
\begin{figure}[H]
\centering
\includegraphics[width=\textwidth]{ModificaMeet/AggiungiInvitati.png}
\caption{aggiunta degli invitati}
\label{aggiunta_degli_invitati}
\end{figure}
\noindent Se si desidera modificare un'email, basta cliccarci sopra per ritornare alla visualizzazione del testo. 
Per eliminarla, è possibile cliccare sulla \textit{X}. 
\\
Sorge un problema: se l'utente digita una sola email e, senza spostare il focus dal \texttt{TextField}, 
clicca sul pulsante \textit{Salva}, l'email digitata non sarà aggiunta come card e quindi non sarà inclusa nell'array
di email da inviare al backend. Per tale motivo, al momento del salvataggio, è necessario controllare l'input 
rimasto all'interno del \texttt{TextField} e, se questo risulta essere un'email valida, aggiungerla all'array di invitati.
\subsubsection{Salvataggio delle modifiche}
Quando si salvano le modifiche, viene chiamata la funzione \texttt{modifyMeet} \ref{funzione_modifyMeet}:
\begin{lstlisting}[language=typescript, frame=lines, basicstyle=\ttfamily\scriptsize, numbers=left, 
    label={funzione_modifyMeet}, caption={funzione modifyMeet}]
const modifyMeet = async () => {

    const payload = {
        evento: isDragged ? modifiedDraggedEvent : eventDetails,
        invitati: inviteesList,
    };

    try {
        const response = await modifyMeeting(payload);
        if (response.status == 200) {
            showPopup('Modifica avvenuta con successo', true);
            
            if (!isDragged) {
                const updatedEvents = events.map((event) =>
                event.id == eventDetails.id ? { ...event, ...eventDetails } : event
            );
            setEvents(updatedEvents);
            }
        } else
...
\end{lstlisting}
\begin{itemize}
    \item \textbf{Righe 3-6}: viene costruito l'oggetto da inviare a backend. Questo include il JSON dell'evento modificato e l'array
    di stringhe contenente le email da aggiungere agli invitati. \texttt{isDragged} è un \texttt{boolean} che indica se l'evento
    è stato modificato dall'intrefaccia del \texttt{Dialog} o se è stato trascinato, in base al suo valore viene passato 
    l'evento corretto.
    
    \item \textbf{Righe 13-17}: se l'operazione di modifica dell'evento dall'interfaccia del \texttt{Dialog} ha
    avuto successo, gli eventi del \texttt{Fullcalendar} vengono aggiornati manualmente in modo da visualizzare la modifica senza che 
    si debba ricaricare la pagina.
\end{itemize}
Inoltre, tutte le persone invitate al meeting ricevono una email di notifica riguardante la modifica apportata.