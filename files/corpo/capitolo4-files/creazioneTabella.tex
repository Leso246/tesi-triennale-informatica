\section{Creazione della tabella a database}
Per garantire un'efficace gestione dei meeting, è stato necessario creare una tabella dedicata nel database aziendale.
Questa tabella è stata progettata per memorizzare tutte le informazioni rilevanti relative ai meeting, le quali
saranno utilizzate da \texttt{FullCalendar} \cite{FullCalendarSite} per permettere la corretta visualizzazione e gestione 
da parte degli utenti.
\\
\\
Per collegare un evento nel database aziendale con il corrispondente meeting su Webex, viene memorizzato l'ID 
univoco assegnato da Webex al momento della creazione del meeting. Inoltre, per ragioni di efficienza e praticità, 
viene salvato anche il link del meeting, generato anch'esso al momento della creazione.
\\
\\
Viene riportata di seguito la query con cui è stata generata la tabella:
\begin{lstlisting}[language=SQL, frame=lines]
CREATE TABLE meet (
    id BIGSERIAL PRIMARY KEY,
    data_inizio TIMESTAMP NOT NULL,
    data_fine TIMESTAMP NOT NULL,
    id_utente BIGINT NOT NULL,
    id_azienda BIGINT NOT NULL,
    link VARCHAR(200) NOT NULL,
    webex_id VARCHAR(200) NOT NULL,
    FOREIGN KEY (id_utente) REFERENCES utente(id_utente),
    FOREIGN KEY (id_azienda) REFERENCES aziende(id_azienda)
);
\end{lstlisting}
\vspace{20pt}
\begin{table}[ht]
    \centering
    \begin{adjustbox}{width=13cm, center}
    \begin{tabular}{|c|c|c|c|c|c|c|}
        \hline
        \rowcolor{gray!30}
        id & data\_inizio & data\_fine & id\_utente & id\_azienda & link & webex\_id \\
        \hline
        50 & 2024-06-28 15:00:00 & 2024-06-28 16:00:00 & 73 & 51 & webex.com\//meet\//ex1 & 93d7d864bd9b4 \\
        \hline
        51 & 2024-07-13 14:30:00 & 2024-07-13 16:30:00 & 34 & 62 & webex.com\//meet\//ex2 & 34421188b307b9\\
        \hline
    \end{tabular}
    \end{adjustbox}
    \caption{Tabella a database di esempio}
\end{table}
\textit{Nota: i timestamp a database vengono automaticamente visualizzati in un formato leggibile da un umano.}
\\
\\
\\
Segue una descrizione dettagliata delle colonne:
\begin{enumerate}
    \item \textbf{id}:
        \begin{itemize}
            \item \textbf{Tipo}: BIGSERIAL
            \item \textbf{Descrizione}: chiave primaria della tabella. È un identificatore univoco generato automaticamente per ogni record.
        \end{itemize}
    \item \textbf{data_inizio}:
        \begin{itemize}
            \item \textbf{Tipo}: TIMESTAMP
            \item \textbf{Descrizione}: indica la data e l'ora di inizio del meeting. Questo campo non può essere nullo.
        \end{itemize}
    \item \textbf{data_fine}:
        \begin{itemize}
            \item \textbf{Tipo}: TIMESTAMP
            \item \textbf{Descrizione}: indica la data e l'ora di fine del meeting. Questo campo non può essere nullo.
        \end{itemize}
    \item \textbf{id_utente}:
        \begin{itemize}
            \item \textbf{Tipo}: BIGINT
            \item \textbf{Descrizione}: identificatore dell'utente associato al meeting. 
            Questo campo è una chiave esterna che fa riferimento alla colonna \texttt{id_utente} della tabella \texttt{utente}. 
            Non può essere nullo.
        \end{itemize}
    \item \textbf{id_azienda}:
        \begin{itemize}
            \item \textbf{Tipo}: BIGINT
            \item \textbf{Descrizione}: identificatore dell'azienda associato al meeting.  \\
            Questo campo è una chiave esterna che fa riferimento alla colonna \texttt{id_azienda} della tabella \texttt{azienda}. 
            Non può essere nullo.
        \end{itemize}
    \item \textbf{link}:
        \begin{itemize}
            \item \textbf{Tipo}: VARCHAR(200)
            \item \textbf{Descrizione}: contiene il link univoco del meeting generato da Webex.
        \end{itemize}
    \item \textbf{webex_id}:
        \begin{itemize}
            \item \textbf{Tipo}: VARCHAR(200)
            \item \textbf{Descrizione}: id univoco del meeting generato da Webex.
            Viene utilizzato per gestire tutte le operazioni che si effettuano sul meeting tramite le API di Webex.
        \end{itemize}
\end{enumerate}
\vspace{20pt}
Dato l'utilizzo di \textbf{JPA}, si è resa necessaria la mappatura della tabella a backend:
\begin{lstlisting}[language=java, frame=lines, basicstyle=\ttfamily\scriptsize, numbers=left]
@Entity
@Table(name = "meet")
@Data
public class Meet {
  @Id
  @GeneratedValue(strategy = GenerationType.IDENTITY)
  @Column(name = "id", nullable = false)
  private long id;
	
  @Column(name = "data_inizio", nullable = false)
  private Timestamp dataInizio;
	
  @Column(name = "data_fine",nullable = false)
  private Timestamp dataFine;
	
  @ManyToOne
  @JoinColumn(name = "id_utente", referencedColumnName = "id_utente", nullable = false)
  private Utente utente;
	
  @ManyToOne
  @JoinColumn(name = "id_azienda", referencedColumnName = "id_azienda", nullable = false)
  private Azienda azienda;
	
  @Column(name = "link", nullable = false)
  private String link;
	
  @Column(name = "webex_id", nullable = false)
  private String webexId;
}
\end{lstlisting}
Segue una breve spiegazione del codice:
\begin{itemize}
    \item \textbf{@Entity}: indica che questa classe è un'entità JPA, mappata a una tabella del database.
    \item \textbf{@Table(name = "meet")}: specifica il nome della tabella a database  a cui è mappata questa entità.
    \item \textbf{@Data}: un'annotazione di Lombock che genera automaticamente getter, setter, toString, equals, e hashCode per la classe.
    \item \textbf{@Id}: definisce il campo id come la chiave primaria della tabella.
    \item \textbf{@GeneratedValue(strategy = GenerationType.IDENTITY)}: delega al database la generazione dei valori della chiave primaria.
    \item \textbf{@Column(name="", nullable=false)}:  definisce il nome della colonna nel database e imposta che il valore del campo non può essere NULL.
    \item \textbf{@ManyToOne}: indica una relazione molti-a-uno, dove ogni Meet è associato a un singolo Utente e una singola Azienda,
    ma ogni Utente e Azienda possono essere associati a più Meet.
    \item \textbf{@JoinColumn(name = "", referencedColumnName = "", nullable = false)}:  specifica la colonna da utilizzare per la join 
    con un'altra entità, definendo il nome della colonna locale e la colonna di riferimento dell'altra entità.
\end{itemize}
Il tipo e il nome definiti sotto ogni annotazione \textbf{@Column} sono quelli utilizzati per gestire l'entità. 
Ad esempio, se si dispone di un oggetto \texttt{Meet} denominato \texttt{meet} e si desidera ottenere il timestamp che 
indica la data di inizio, è sufficiente invocare \texttt{meet.getDataInizio()}. 
Questo illustra la potenza di JPA, che semplifica l'accesso e la manipolazione degli attributi delle entità.