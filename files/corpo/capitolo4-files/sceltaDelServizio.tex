\section{Scelta del servizio}
Come riportato \hyperref[sec:scelta_del_servizo]{precedentemente}, il primo passo è stato selezionare il 
servizio di terze parti da utilizzare per effettuare le chiamate online.
\\
Le possibili opzioni considerate sono state:
\begin{itemize}
    \item Cisco Webex
    \item Google Meet
    \item Zoom
\end{itemize}
Dopo un'attenta lettura della documentazione e delle funzionalità offerte dai vari servizi,
la scelta è ricaduta su \textbf{Cisco Webex} per i seguenti motivi:
\begin{itemize}

    \item \textbf{Facilità di integrazione}: Cisco Webex consente di integrare facilmente la propria applicazione nel loro sito, 
    permettendo di abilitare tutte le funzionalità necessarie in modo comodo e rapido.

    \item \textbf{Gestione tramite API RESTful}: la gestione dei meeting avviene interamente attraverso API RESTful, 
    le quali possono essere chiamate da qualsiasi ambiente, garantendo una notevole flessibilità.

    \item \textbf{Completezza delle API}: Cisco Webex offre una vasta gamma di API per soddisfare qualsiasi esigenza, 
    ognuna delle quali è altamente personalizzabile in base alle specifiche necessità del progetto.

    \item \textbf{Documentazione chiara ed efficace}: la documentazione fornita da Cisco è ben strutturata e di facile comprensione, facilitando il processo di sviluppo.
    
    \item \textbf{Ambiente di prova delle API}: Cisco Webex offre la possibilità di testare le API in un ambiente dedicato senza la necessità di effettuare il login, utilizzando esempi predefiniti o personalizzati.

    \item \textbf{Supporto rapido ed efficiente}: in caso di necessità, il supporto fornito da Cisco è tempestivo e competente.
\end{itemize}