\subsection{Backend}
Il recupero dei meeting nel backend viene gestita tramite una chiamata 
GET all'API \texttt{\//getMeet} con il parametro \texttt{jwtToken}.
% File del controller
\subfile{backend/controller.tex}
\clearpage
% File del service
\subfile{backend/service.tex}
% File del Repository
\subfile{backend/repository.tex}
\subsubsection{Recupero degli invitati}
\label{sec:recupero_invitati_backend}
In relazione al recupero degli invitati a backend, si pone questo problema: recuperando le mail di tutti
gli invitati al meet da Webex, viene recuperato anche il contatto del candidato. Se questo venisse mostrato 
a frontend, verrebbe meno l'obiettivo prefissato, ovvero non consentire alle aziende di contattare i candidati
al di fuori della piattaforma, almeno fino al primo colloquio. Per risolvere questo problema, una volta 
ottenuto l'array contenente la mail degli invitati, viene recuperata l'istanza \texttt{Utente} del candidato
associato al meet grazie al metodo \texttt{findMeetById}, e la sua mail viene sostituita con il suo nome utente.
\begin{lstlisting}[language=java, frame=lines, basicstyle=\ttfamily\scriptsize, numbers=left]
private static ArrayList<String> removeRisumaUserEmail(
  ObjectNode obj, 
  MeetJPA repo, 
  ArrayList<String> invitees) {
	try {
		Long meetId = obj.get("id").asLong();
		Utente utente = repo.findMeetById(meetId).getUtente();
		String emailToRemove = utente.getEmail();
		String username = utente.getUsername();
		
		invitees.remove(emailToRemove);
		invitees.add(username);
			
		return invitees;
	} catch (Exception e) {
		e.printStackTrace();
		return null;
	}
}
\end{lstlisting}
