\subsubsection{Repository}
\begin{lstlisting}[language=java, frame=lines, basicstyle=\ttfamily\scriptsize, numbers=left]
@Repository
public interface MeetJPA extends JpaRepository<Meet, Long> {
    public Meet findMeetById(Long id);

    @Query(value = "select new dto.MeetDTO("
        + "m.id, "
        + "m.dataInizio, "
        + "m.dataFine, "
        + "m.utente.username, "
        + "m.azienda.anagraficaAz.nome, "
        + "m.link, "
        + "m.webexId) "
        + "from Meet m  "
        + "where m.azienda.id =:id_azienda")
    public List<MeetDTO> getMeetAzienda(Long id_azienda);
    
    @Query(value = "select new dto.MeetDTO("
        + "m.id, "
        + "m.dataInizio, "
        + "m.dataFine, "
        + "m.utente.username, "
        + "m.azienda.anagraficaAz.nome, "
        + "m.link, "
        + "m.webexId) "
        + "from Meet m  "
        + "where m.utente.id =:id_utente")
    public List<MeetDTO> getMeetUtente(Long id_utente);
}
\end{lstlisting}
\begin{itemize}
    \item \textbf{Riga 1}: questa annotazione indica che l'interfaccia è una repository Spring. La classe fornisce operazioni CRUD su un oggetto. 
    \cite{RepositorySpring}

    \item \textbf{Riga 3}: questo metodo consente di recuperare un'istanza di \texttt{Meet} a database dato il suo id. Spring Data JPA
    genera automaticamente l'implementazione di questo metodo basandosi sul suo nome. Verrà utilizzato per il 
    \hyperref[sec:recupero_invitati_backend]{recupero degli invitati}.

    \item \textbf{Righe 4-26}: queste due query personalizzate recuperano tutti i meet associati all'ID dell'utente, sono un esempio di utilizzo
    del linguaggio \textbf{JPQL}. 
\end{itemize}
È interessante notare l'uso della sintassi \texttt{new dto.MeetDTO} nelle query JPQL: è stato necessario creare una classe \texttt{MeetDTO} in quanto, 
se si fosse recuperata l'intera istanza di \texttt{Meet}, il database avrebbe restituito come attributi di \texttt{utente} e di \texttt{azienda} le 
due istanze nella loro interezza, al posto dei due soli nomi, che sono invece di nostro interesse. Ovviamente ciò avrebbe fatto sorgere un 
problema di sicurezza, in quanto sarebbero stati restituiti a frontend dati sensibili riguardo le due utenze.
\\
\\
DTO, acronimo di Data Transfer Object, rappresenta un oggetto progettato appositamente per trasferire dati tra processi 
all'interno di un sistema, in questo caso tra il database e il backend \cite{DataTransferObject}. L'implementazione di \texttt{MeetDTO} consente di selezionare 
solo i campi pertinenti da \texttt{Meet} necessari per l'interesse dell'applicazione, escludendo informazioni non richieste e sensibili.
Questo approccio non solo migliora le prestazioni dell'applicazione evitando il trasporto di dati non necessari, 
ma anche rafforza la sicurezza limitando l'esposizione di informazioni sensibili.
\begin{lstlisting}[language=java, frame=lines, basicstyle=\ttfamily\scriptsize, numbers=left]
@Data
@NoArgsConstructor
public class MeetDTO {
    private long id;
    private Timestamp start;
    private Timestamp end;
    private String nomeUtente;
    private String nomeAzienda;
    private String link;
    private String webexId;
        
    public MeetDTO(
        long id, 
        Timestamp start, 
        Timestamp end, 
        String nomeUtente, 
        String nomeAzienda, 
        String link,
        String webexId) {
            
        super();
        this.id = id;
        this.start = start;
        this.end = end;
        this.nomeUtente = nomeUtente;
        this.nomeAzienda = nomeAzienda;
        this.link = link;
        this.webexId = webexId;
    }       
}
\end{lstlisting}
Quando vengono recuperati i vari attributi della query JPQL, 
viene utilizzato il costruttore della classe \texttt{MeetDTO} per creare un 
oggetto di questa classe che viene poi aggiunto alla \texttt{List} che il metoodo restituisce.