\subsection{Backend}
La creazione di un meeting nel backend viene gestita tramite una chiamata POST all'API \texttt{\//addMeet} con il seguente body:
\begin{lstlisting}[language=json,firstnumber=1]
{
    "data_inizio": "2024-06-25T16:00:00.000Z", 
    "data_fine": "2024-06-25T17:00:00.000Z",   
    "azienda": "techSPA@gmail.com", 
    "utente": "mario@gmail.com",
    "invitati": [azienda1@outlook.com, azienda2@virgilio.it]
}
\end{lstlisting}
\begin{itemize}
    \item \textbf{azienda}: l'email con cui l'azienda si è registrata su RisUma.
    \item \textbf{utente}: l'email con cui l'utente si è registrata su RisUma.
    \item \textbf{invitati}: un array contenente le email di ulteriori invitati, nel caso più persone dell'azienda volessero 
    partecipare al colloquio. Questo attributo è opzionale; può essere vuoto o assente, e la creazione del meeting andrà comunque a buon fine.
\end{itemize}
% File del controller
\subfile{backend/controller.tex}
% File del service
\subfile{backend/service.tex}
% File di CreateMeeting
\subfile{backend/utilita.tex}