\subsubsection{Controller}
\begin{lstlisting}[language=java, frame=lines, basicstyle=\ttfamily\scriptsize, numbers=left]
@PostMapping(value = "/addMeet", consumes = "application/json")
public ResponseEntity<?> addMeeting(@RequestBody ObjectNode obj) {
  try {
	Mono<Boolean> resultMono = serviceMeet.addMeet(obj);
	Boolean result = resultMono.toFuture().get();

	if (result) {
	  return new ResponseEntity<>(result, HttpStatus.OK);
	} else {
	  return new ResponseEntity<>(result, HttpStatus.INTERNAL_SERVER_ERROR);
	}
  } catch (Exception e) {
	e.printStackTrace();
	return new ResponseEntity<>(e.getMessage(), HttpStatus.INTERNAL_SERVER_ERROR);
  }
}    
\end{lstlisting}
\begin{itemize}
    \item \textbf{Riga 1}:
    l'annotazione indica che il metodo risponde a richieste HTTP POST sull'URL /addMeet 
    e che il body della richiesta deve essere in formato JSON.
  
    \item \textbf{Riga 2}:
    questo è il metodo che gestisce la richiesta HTTP. 
        \begin{itemize}
            \item \textbf{ResponseEntity\textless{}?{}\textgreater{}}: \texttt{ResponseEntity} è una classe di Spring che 
            rappresenta l'intera risposta HTTP inviata al client. Il tipo generico "\textless{}?\textgreater{}" indica che il corpo 
            della risposta può essere di qualsiasi tipo. In altre parole, il metodo può restituire una risposta che può 
            contenere dati di qualsiasi tipo.
            
            \item \textbf{@RequestBody ObjectNode obj}: è un'annotazione di Spring che indica al framework di 
            utlizzare il corpo della richiesta HTTP per popolare \texttt{obj}, \texttt{ObjectNode} è un oggetto fornito 
            dalla libreria \textbf{Jackson} utile a rappresentare JSON in Java.
            
        \end{itemize}
  
    \item \textbf{Riga 4}:
    all'interno del blocco try, viene chiamato il \textbf{service} attraverso
    \texttt{serviceMeet.addMeet(obj)}, che restituisce un oggetto 
    \texttt{Mono\textless{}Boolean\textgreater{}}. 
    %TODO La classe \texttt{Mono} è parte del progetto Reactor, utilizzato per la programmazione reattiva in Java.
  
    \item \textbf{Riga 5}:
    il risultato del Mono viene convertito in un Future e poi ottenuto. Questo metodo blocca l'esecuzione 
    finché il risultato non è disponibile.
  
    \item \textbf{Riga 7}:
    se il risultato è \texttt{true}, viene restituita una risposta HTTP con lo stato OK (200), altrimenti
    viene restituita una risposta con lo stato INTERNAL_SERVER_ERROR (500).
  
    \item \textbf{Riga 12}:
    se si verifica un'eccezione durante l'esecuzione, l'eccezione viene stampata nello stack trace e viene 
    restituita una risposta con lo stato INTERNAL_SERVER_ERROR (500), contenente il messaggio dell'eccezione.
  \end{itemize}